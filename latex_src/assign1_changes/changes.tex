\documentclass[]{article}

\usepackage{sectsty}

\usepackage[parfill]{parskip}

\usepackage{hyperref}

\usepackage[T1]{fontenc}

\usepackage[T1]{fontenc}

\sectionfont{\fontsize{10}{10}\selectfont}


\begin{document}


\author{
  Haroun Habeeb\\
  \texttt{2013CS10225}
  \and
  Kabir Chhabra\\
  \texttt{2013CS50287}
  \and
  Harman Kumar\\
  \texttt{2013CS10224}
}

\title{Changes}
\maketitle

\begin{abstract}
\noindent This document enumerates the changes with respect ot the previous design document. It is segregated by section.
\end{abstract}

\section{ \bf Threading :}
Earlier the implementation followed a barrier synchronization model. We have modified it to now use a One-to-One Synchronization model. This was done using a message queue, which we have called mailBox.

\section{ \bf User Interface : }
Apart from the key-mappings that were intended, we have provided the ability to zoom within certain limits. We have also implemented a Graphical User Interface, using the GLUI library. The GLUI allows you to control the physical parameters, rotation, fullscreen mode. The GLUI window is available only in the windowed mode.

\section{ \bf Physics Used :}
Support for gravity along the Y-axis was provided. The value of Coefficient of restitution (initially set to 1 for completely elastic collisions) can be changed to a value between 0 and 1 to simulate collisions that are closer to the real-life model.

\section{ \bf Graphics :}
A background has been provided. The background is a texture which is loaded using SDL and other functions provided by glut .The table/cube is drawn as a wire frame. The balls have lighting effects.

\section{ \bf Project Structure :}
There have been revamps in our project structure. Comments describing the role of the functions and variables were written, modularity was incorporated and object oriented programming paradigms were embodied.
\end{document}